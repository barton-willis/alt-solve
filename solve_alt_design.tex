\documentclass[]{scrartcl}
\usepackage[letterpaper,margin=1.0in]{geometry}

\usepackage[activate={true,nocompatibility},final,tracking=true,kerning=true,spacing=true,factor=1100,stretch=10,shrink=10]{microtype}
\microtypecontext{spacing=nonfrench}
\usepackage[american]{babel}
\usepackage{mathptmx}
\usepackage{enumitem}
\usepackage{xcolor}
\usepackage{upgreek}

\newlist{alphalist}{enumerate}{3}
\setlist[alphalist]{label=\alph*),before=\raggedright}

\newcommand{\altsolve}{\texttt{solve\_alt}}
\newcommand{\solve}{\texttt{solve}}
\newcommand{\algsys}{\texttt{algsys}}
\newcommand{\solveexplicit}{\texttt{solveexplicit}}
\newcommand{\fpprec}{\texttt{fpprec}}

\newcommand{\multivaluedinverse}{\texttt{multivalued\_inverse}}
\newcommand{\singlevaluedinverse}{\texttt{single\_valued\_inverse}}
\newcommand{\solveinversepackage}{\texttt{solve\_inverse\_package}}
\title{An alternative solve function for Maxima}
\author{Barton Willis}
\begin{document}

\maketitle

\section{Aims}


The \altsolve\/ package aims to be a replacement for Maxima's solve function. The goals are

\begin{alphalist}[]

\item to support the needs for a variety users, including high school students, university students, mathematicians, engineers, statisticians, and scientists,

\item to provide clear user documentation that is centered  on user's questions,

\item to focus on symbolic solution to equations that are pertinent to science and engineering problems,

\item to find all real and complex solutions to equations, but take care to not return spurious solutions.

\end{alphalist}
Additionally, \altsolve\/  aims to be compatible with Maxima's current solve function. For compatibilty, we require:

\begin{alphalist}[]

\item Solutions are expressed as Maxima lists with solutions that invovle objects that Maxima's core  understands.

\item We make an effort to support most the option variables that directly affect solve.

 \item The \altsolve\/ code requires only modest modifications to Maxima's core.

\item Except for algebraically equivalent, but different results, \altsolve\/  runs the testsuite and the share testsuite with no errors and in roughy the same time as the standard Maxima.

\item The code runs under all Common Lisp versions that are supported by Maxima.
\end{alphalist}
We do \emph{not} aim
\begin{alphalist}[]

\item to solve inequalities, Diophantine equations, differential equations, recursion relations, or to dispatch purely numerical methods (the Newton method, for example),

\item to build a platform that could be used to display the steps done to find the solution.

\end{alphalist}

Our development priorities are (in order from highest to least)

\begin{alphalist}[]

\item to write  Common Lisp source codethat is easy to extend and to fix,

\item to have a modest code size,

\item to be conservative with memory usage,

\item to solve equations quickly.

\end{alphalist}

\section{Top-level functions}

The top-level functions are \texttt{solve}, \texttt{ssolve}, \texttt{solve\_filter},
and \texttt{linsolve}. The function \texttt{solve} takes at one or two arguments. The first
argument gives the equations to be solved. The first argument is either a Maxima expression,
a list of Maxima expressions, or a set of Maxima expressions. If any of these expressions is
not explicitly of the form \(a = b\), the expression is assumed to vanish.

When given, the second argument gives the unknowns. The unknowns can either be a list, set, or
a single unknown. Duplicate unknowns are expunged. If not specified, the unknowns defaults to
the set of all nonconstant variables in the first argument.

For the case of solving for a single unknown \(x\), the function \texttt{solve} returns a list
of solutions of the form \(x = a\), where the expresion \(a\) does not explicitly depend on \(x\).
Any declared dependence (using the \texttt{depends} mechanism) is ignored by \texttt{solve}.
The order of the solutions is unspecified. When solving for two or more unknowns, the solution
is a list of lists of the form \([x_1 = a_1, x_2 = a_2, \cdots x_n = a_n] \). The ordering of
this lists is unspecified.

When \texttt{solve} is unable to find a solution and the option variable \texttt{solveexplicit}
is true (default false), Maxima prints a message saying that there is no method for solving the
equation and it returns the empty list. When \texttt{solveexplicit} is false and there is no
method for solving, \texttt{solve} solve chooses an unspecified subexpression of the equation
and solves for it; for example

\begin{verbatim}
   (%i1)	block([solveexplicit : true], solve(bessel_j(0,x) = x,x));
   Solve: No method for solving x-bessel_j(0,x) for x; returning the empty list.
   (%o1)	[]

   (%i2)	block([solveexplicit : false], solve(bessel_j(0,x) = x,x));
   Solve: No method for solving x-bessel_j(0,x) for x; returning an implicit solution.
   (%o2)	[x=bessel_j(0,x)]
\end{verbatim}
\paragraph{Solving for non mapatoms} Generally, each unknown should be a mapatom, but when an unknown is not a mapatom, a new unique symbol is created for the unknown. Using \texttt{ratsimp}, this new symbol is substituted into the equation.  After that, the equation is solved for the new symbol. Finally, the orginal expression is substituted for the new symbol; for example
\begin{verbatim}
  (%i2)	solve(x^2 +x + 1,x^2);
  (%o2)	[x^2=-x-1]

  (%i3)	solve(x^4+x^2+1,x^2);
  (%o3)	[x^2=-(sqrt(3)*%i+1)/2,x^2=(sqrt(3)*%i-1)/2]
\end{verbatim}

\section{Representation of solutions}

To be compatible with Maxima, solutions to equations with one variable have the form of a Maxima
list; thus a solution has the form \([x = e_1, x = e_2, \cdots, x = e_n]\), where the
expressions \(e_1, e_2, \dots, e_n\) are free of \(x\). The order of solutions is
\emph{unspecified}. The list structure of a solution allows for the following example to
yield true:

\begin{verbatim}
(%i1)	eq : (x-5)*(x-2)=0$
(%i2)	every(map(lambda([q], subst(q, eq)), solve(eq,x)));
(%o2)	true
\end{verbatim}

The solutions should involve objects that Maxima's core functions such as simplification,
sign, mapatom, featurep, and declare understand. Although it's tempting to, for example,
invent objects such as \(x = (1 ... \infty) \uppi\) to represent \(\uppi\) times a nonnegative
integer, but without modifing Maxima's simplifer for sine,
\(\sin \left ((1 ... \infty) \uppi \right) \) would not simplify  to zero. This is undesirble.
Thus we attempt to represent solutions as standard Maxima objects.

Mapatoms of the form \%zk, \%rk, and \%ck, where k is a nonnegative integer, represent arbitrary
integers, real numbers, and complex numbers, respectively. Here is an example of a solution that
involves and arbitrary integer:
\begin{verbatim}
  (%i2)	solve(sin(x)=0,x);
  (%o2)	[x=%pi*%z1]
\end{verbatim}
The fact that \%z1 is a declared integer is automatically entered into the fact database into the
context `initial.'

\section{Multiplicities}

The global variable \texttt{multiplicities} is either a Maxima list of the multiplicities of the solutions for the most recently solved equation or it is the symbol \texttt{not\_set\_yet}. For
a system of equations, no effort is made to determine the multiplicity. For equations of
two or more varibles, the value of \texttt{multiplicities} is \texttt{not\_set\_yet}.

Any function that sorts or reverses a solution set must adjust the multiplicities. When the
multiplicity isn't a positive integer, the multiplicity is \emph{unspecified}. Thus although
arguably the following example might be suboptimal, it isn't a bug:
\begin{verbatim}
  (%i4)	solve(abs(x-2),x);
  (%o4)	[x=2]

  (%i5)	multiplicities;
  (%o5)	[1]
\end{verbatim}
For definite integration of meriomorphic functions, Maxima uses the multiplicities data. Thus
it is importate to correctly multiplicities for meriomorphic functions, but for other cases,
it is less important.

\section{Use of database facts}

Type declarations on the unknown do not limit the solution set to include only the declared type.
For example if \(x\) is declared to be an integer, a solution set for an equation involving \(x\)
may include noninteger values. Within the solve process, a type declaration might be used during
simplification. So it is possible that a type declaration for an unknown will alter the solution
set, but again, the solution set is free to contain members that do not satisfy the declared types
(if any) of the unknowns.

Setting the option variable \texttt{domain} to real (its default) does not limit the solution set
to contain only real values. Again, internal to the solve process, the value of \texttt{domain}
might affect simplications--so it is possible that the value of \texttt{domain} will affect a
solution set.

Similarly, inequality assumptions on the unknown (for example assume( \(0 < x, x < 1\)) do not
limit the solution set to values that satisfy the assumptions in the fact database. But again,
simplifications that are done in the solve process may use the fact database.

An experimental function \texttt{solve\_filter} attemps to filter out solutions that are
inconsistent with the fact database. Especially for deciding if a solution is real,
the \texttt{featurep} based mechanism is weak.

\begin{verbatim}
  (%i2)	declare(x, real)$

  (%i3)	solve_filter(x^3 + x + 9=0,x);
  (%o3)	[x=-(((3*(sqrt(6573)+81))/2)^(1/3)-3/((3*(sqrt(6573)+81))/2)^(1/3))/3]

  (%i4)	solve_filter(x^2 + x + a = 0,x);
  (%o4)	[]

  (%i5)	assume(0<a, a<1/4)$

  (%i6)	solve_filter(x^2 + x+ a,x);
  (%o6)	[x=(sqrt(1-4*a)-1)/2,x=-(sqrt(1-4*a)+1)/2]
\end{verbatim}

\section{Solve is an ordinary function}

The function \solve \/ evaluates its input. That makes \solve \/ an \emph{ordinary} Maxima function.
This means, for example, that the function call \(\mathrm{solve}(x/x=1,x)\) is effectively evaluated to
\(\mathrm{solve}(1=1,x)\) before \solve \/ starts its work. Efforts to thwart this
(by setting simp to false or by quoting) are likely to be unsuccessful.


\section{Use of asksign}

The solve function sparingly uses \texttt{asksign}. And when it does, these facts are placed
into a context that is killed once solve terminates. An example of using asksign:

\begin{verbatim}
  (%i2)	solve(abs(x)=a,x);
  "Is "a" positive, negative or zero?" pos;
  (%o2)	[x=a,x=-a]

  (%i3)	solve(abs(x)=a,x);
  "Is "a" positive, negative or zero?"negative;
  (%o3)	[]

  (%i4)	solve(abs(x)=a,x);
  "Is "a" positive, negative or zero?"zero;
  (%o4)	[x=0]
\end{verbatim}
Adhering to the requirement that solutions are Maxima lists, not conditionalized expressions,
for this example, the alternative to using asksign is to decline to solve the equation. In
such cases, \texttt{solve} uses \texttt{asksign}.

\section{Option variables}

For compatibilty, the goal is to support most of the option variables that directly affect
the solve process. These are:
\begin{description}

\item [algexact] For now, \altsolve\/ does not implement the algexact option variable.

\item [breakup] For now, \altsolve\/ does not implement the breakup option variable.

\item [globalsolve] For now, \altsolve\/ does not implement the globalsolve option variable.

\item [programmode] For now, \altsolve\/ does not implement the programmode option variable.

\item [solveradcan] When solveradcan is true, \altsolve\/ applies radcan early in the solve process.
      Exactly what ``early'' means is unspecified.

\item [solvefactors] For now, \altsolve\/ ignores this option variable.

\item [solvenullwarn] When this option variable is true and either the equation list or the unknown list is
empty, print a warning.

\item [solvetrigwarn] For now, \altsolve\/ does not implement the solvetrigwarn option variable.

\item [solvedecomposes] When solvedecomposes is true, the function \texttt{polydecom} is applied to
polynomial equations of degree five or more. When the decomposition yields polynomials of degree four
or less, \altsolve\/ returns a solution in terms of radicals. There is, I think, little reason to
set solvedecomposes to false.

\item [solveexplicit] (default false) When \altsolve\/  is unable to find a solution and the option
variable \solveexplicit\/ is false, return an equivalent, but possibly simplified equation;
otherwise print a message that there is no method for solving the equation and return the empty list.

\item [linsolve\_params] (default true) When linsolve\_params is true, linsolve introduces \%r
parameters for underdetermined equations; when false, linsolve expresses some unknowns in terms
of others. Which variables are explicitly solved for is unspecified.

For now, \altsolve \/ ignores this option variable.

\item [use\_to\_poly] (default true)  When use\_to\_poly is true, \altsolve\/ is able to solve
more algebraic equations.  Likely there is little reason to set this option variable to false.

\item[solve\_ignores\_conditions] (default false) When this option variable is false, \altsolve\/
attempts to remove solutions that are not in the natural domain of the equation; when true,
solutions may include those that are outside the domain; for example
\begin{verbatim}
(%i7)	block([solve_ignores_conditions : true], solve( x * (log(x) - 1)  = 0));
(%o7)	[x=%e,x=0]

(%i8)	block([solve_ignores_conditions : false], solve( x * (log(x) - 1)  = 0));
(%o8)	[x=%e]
\end{verbatim}

The correct functioning of this mechanism requires the correct functioning of many other Maxima components. But it is subject various limitations.

\item[solve\_inverse\_package] When solve\_inverse\_package is multivalued\_inverse, \altsolve\/
attemps to find all solutions (both real and complex) to equations; when its value is
single\_valued\_inverse, the effort is to find one or more soltions, but primarly a real solution.
Examples:
\begin{verbatim}
  (%i1)	block([solve_inverse_package : multivalued_inverse], solve(1.03^x = 5.2));
  (%o1)	[x=33.83087013568215*(2*%i*%pi*%z3+1.648658625587382)]

  (%i2)	block([solve_inverse_package : single_valued_inverse], solve(1.03^x = 5.2));
  (%o2)	[x=55.77555586031892]
\end{verbatim}
For a problem in finiancial math or high school level math, likely is the one that is desired. Another
example is
\begin{verbatim}
(%i1)	block([solve_inverse_package : multivalued_inverse], solve(sin(x) = 1/2));
(%o1)	[x=(12*%pi*%z5+%pi)/6,x=(12*%pi*%z6+5*%pi)/6]

(%i2)	block([solve_inverse_package : single_valued_inverse], solve(sin(x) = 1/2));
(%o2)	[x=%pi/6]
\end{verbatim}

Maxima's integration, especially definite integration code, needs to solve equations. But
this code would mostly misbehave using the  multivalued\_inverse package. The \altsolve\/
code automatically sets solve\_inverse\_package to single\_valued\_inverse when inside
the integration code.

\end{description}
%I'm not sure what solve(0=0, []) should return. Returning the symbol  \all makes a bit of a mess (a solution cannot be relied on to be a list)








Solutions to equations with two or more variables are lists of lists of the form \([x_1 = e_1, x_2 = e_2,
\cdots, x_n = e_n] \), where the expressions \(e_1\) through \(e_n\) are free of the symbols \(x_1\)
through \(x_n \). The order of variables in each solution is unspecified, and the order of the lists is
unspecified.

\section{Keep float mechanism}

The top level solve function encloses all floats (Binary64) and big floats in labeled boxes.
The floats are converted to exact rational form. After solve exits, numbers are converted back
to floats (big floats are converted using the \emph{current} value of \fpprec.

It can, of course, happen that the simplification after conversion back to floats can
involve subtractive cancellation. In such cases, purely numerical methods might be better.

\section{Solve function hierarchy}

 The top level function solve function:
 \begin{alphalist}[]
   \item resets multiplicities to its default value,
   \item expunges constant variables and duplicates from the list of variables,
   \item checks for inequalities and signals an error when found,
   \item creates a new super context--all assumptions made during solve are removed after exiting,
   \item protects all floats and big floats in labeled boxes,
   \item does gensym substitutions when solving for nonatoms,
   \item dispatches the appropriate lower level solve function,
   \item reverts gensym substitutions for nonatom solve,
   \item unboxes the floats and big floats,
   \item kills the super context.
\end{alphalist}

The lower level solve functions are solve-single-equation (solve one equation in one unknown),
redundant-equation-solve (solve two or more equations for one unknown), and solve-multiple-equations
(solve two or more equations in two or more unknowns).

\paragraph{solve-single-equation} If the input is a polynomial, dispatch the polynomial solve code.

First the equation is processed using
The primary method for solving one equation in one unknown is to match the equation to the form \(a F(X) - b = 0\), where \(X\) depends on the unknown, the function \(F\) has a known inverse, and \(a\) and \(b\) are free of the unknown \(x\). When the match is successful, the new tasks is to solve \(X = F^{-1}(b/a) \). In general, \(F^{-1}\) is multi-valued--in such cases, the task is to solve finitely many equations of the form  \(X = F^{-1}(b/a) \).

The multi inverses for functions are stored in a Common Lisp hashtable. To accommodate the needs of diverse users,
there are two hashtables; one is named \multivaluedinverse\/ and the other \singlevaluedinverse.  The option
variable \solveinversepackage\/ is the value of the current hashtable for the function multi inverses.

\emph{I think I want to rename these hashtables to something like \texttt{real\_valued\_inverse}
and \texttt{complex\_valued\_inverse}. Plus there will be a behind arctrig compatibilty hashtable that is primarly for the non-userlevel solve function.}

\section{Polynomial zeros}

The zeros of polynomials of degree four or less are expressed in terms of radicals. Also  cyclotomic
polynomials of degree six are solved in terms of radicals, as well as polynomial of degree five or
greater that decompose into polynomials of degree four or less. \emph{Well there option variable solvedecomp controls using polynomial decomposition, but I think this option could be removed.}

\begin{alphalist}[]

\item  Polynomial coefficients that are not simplified to zero by the  general simplifier are
assumed to be nonzero. Even if an atom has been declared to be zero, the general simplifier does
not simplify  it to zero. Thus even if the atom \(a\) has been declared to be zero,
the solution to \(a x = b\) is \(x = -b/a \).



\end{alphalist}



\end{document}
